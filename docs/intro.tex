\documentclass[letterpaper,12pt]{article}
%\documentclass[oneside,numbers,spanish]{ezthesis}
\usepackage[utf8]{inputenc} % Soporte para acentos
\usepackage[T1]{fontenc}    
\usepackage[spanish,mexico]{babel} % Español
\usepackage[pdftex]{graphicx}
\graphicspath{{Images/}}
\usepackage{amsmath}		% Soporte de símbolos adicionales (matemáticas)
\usepackage{amssymb}		
\usepackage{amsfonts}
\usepackage{latexsym}
\usepackage{booktabs}
\usepackage{longtable}
\usepackage{epstopdf} % Para inserción de imagenes
\usepackage{subfigure}
\usepackage{color}
\usepackage{enumerate}

\usepackage{hyperref}
\usepackage[small]{caption}
\usepackage{tabularx} % Algunos comandos y ambientes para tablas
\usepackage{cite} % Para citas bibliograficas
\usepackage{multicol}
%\usepackage[lmargin=3cm,rmargin=3cm,top=2cm,bottom=2cm]{geometry}
\usepackage[a4paper,width=150mm,top=25mm,bottom=25mm]{geometry}
\parskip=3mm % Indicamos una separación entre los párrafos
%\usepackage{fancyhdr}
%\pagestyle{fancy}
\usepackage{float}
\usepackage{algorithm}
\usepackage[noend]{algpseudocode}

%\usepackage{xr}
%\externaldocument{chapter5}


%\head[]{\itshape}

 \begin{document}
 
\section*{1. Introducción y objetivos}

Bajo la perspectiva y enfoque de la teoría de los Sistemas Complejos, el modelo del consumidor Postkeynessiano  presenta características interesantes debido a que este enfoque de la teoría económica, intenta comprender las propiedades de un sistema económico de consumo, producidas a partir de las características individuales de muchos agentes y de sus interacciones a través del espacio y el tiempo.

Como se presenta a detalle en [Vite R., Carreón G.], los elementos de la teoría de consumo Postkeynessiana, hacen una invitación al Modelado Basado en Agentes debido a que la toma de decisiones de consumo de cada agente económico, se ve afectada por la influencia de su entorno.

El modelo basado en agentes  presentado en la referencia mencionada, tiene como hipótesis de referencia que ``\textit{el consumo depende del ingreso corriente, de las costumbres, de los hábitos, de la moda y de la imitación}''

Este modelo describe la dinámica creada de principalmente dos elementos: el \textbf{consumo por moda} como la influencia sobre el consumo debido a la presencia  de determinados vecinos, y el consumo por imitación al tipo de consumo que se hace debido a cambios en el ingreso (el cual determina la clase social a la que pertenece).

Para modelar las prácticas de consumo por moda, se define un vector de consumo $V_c=V_c(x,y,z)$ que determina el nivel de consumo y clasifica al agente en distintas clases de acuerdo a los valores de las variables binarias $x,y,z$. Este vector de consumo se evoluciona a partir de reglas de actualización a partir de la información local (que es determinada en este caso, por la vecindad de Moore del agente)

Las prácticas de consumo por imitación se modelan a partir de la adaptación del consumo, cuando el agente cambia de ingreso y transita entre clases sociales.  La transición entre clases sociales se hace a partir de un sencillo modelo probabilístico que asigna la probabilidad e transitar de una clase a otra.

Es en esta parte del modelo, en donde nos planteamos los \textbf{objetivos} para este pequeño trabajo de investigación. Nuestra propuesta es modificar el modelado del consumo por imitación y hacer un modelo no probabilístico para la transición entre clases sociales. La hipótesis a incluir está basada en la idea de que ``el consumo de un agente, es el ingreso de otro agente''. así,  intentaremos modelar el ingreso del agente a  partir de la información local que a su vez, influirá en las practicas de consumo por moda, sustituyendo las probabilidades de transición entre clases, por umbrales que definan la clase a la que pertenece el agente. 



\end{document}
